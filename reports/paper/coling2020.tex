\documentclass[11pt]{article}
\usepackage{coling2020}
\usepackage{times}
\usepackage{url}
\usepackage{latexsym}
\usepackage[super]{nth}
\usepackage{color,soul}
\usepackage{multirow}
\usepackage{makecell}
\usepackage{textcomp}
\usepackage[toc,page]{appendix}
\usepackage{graphicx}

\colingfinalcopy

\title{Lexicographical Word-Sense Disambiguation}

\author{Salih Furkan Akkurt \\
  Boğaziçi University \\
  Department of Computer Engineering \\
  34342 Bebek, Istanbul, Turkey \\
  {\tt furkan.akkurt@boun.edu.tr} \\}

\date{}

\begin{document}
\maketitle
\begin{abstract}

\end{abstract}

\section{Introduction}

Dictionaries are great sources of information related to languages.
They contain a lot of information regarding meanings (senses) of words.
As language evolves, senses of words change and dictionaries need to be updated accordingly.
Dictionary writers constantly work on manually detecting semantic changes in current usage.
This is a very time-consuming and expensive process.

In this work, I propose a method to automatically detect semantic changes in words by leveraging the contextualizing power of deep language models.
I use BERT~\cite{devlin-etal-2019-bert} to contextualize words in a corpus and compare their contextualized representations to detect semantic changes.

\blfootnote{
    \hspace{-0.65cm}
    This work is licensed under a Creative Commons
    Attribution 4.0 International Licence.
    Licence details: \url{http://creativecommons.org/licenses/by/4.0/}.
}

\section{System Description}

\section{Experimental Setup}

\section{Results and Discussion}

\subsection{Discussion}

\section{Conclusion}

\bibliographystyle{coling}
\bibliography{coling2020}

\end{document}
